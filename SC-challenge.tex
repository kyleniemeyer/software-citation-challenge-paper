% v2-acmsmall-sample.tex, dated March 6 2012
% This is a sample file for ACM small trim journals
%
% Compilation using 'acmsmall.cls' - version 1.3 (March 2012), Aptara Inc.
% (c) 2010 Association for Computing Machinery (ACM)
%
% Questions/Suggestions/Feedback should be addressed to => "acmtexsupport@aptaracorp.com".
% Users can also go through the FAQs available on the journal's submission webpage.
%
% Steps to compile: latex, bibtex, latex latex
%
% For tracking purposes => this is v1.3 - March 2012

\documentclass[prodmode,acmjdiq]{acmsmall} % Aptara syntax

% Package to generate and customize Algorithm as per ACM style
\usepackage[ruled]{algorithm2e}
\renewcommand{\algorithmcfname}{ALGORITHM}
\SetAlFnt{\small}
\SetAlCapFnt{\small}
\SetAlCapNameFnt{\small}
\SetAlCapHSkip{0pt}
\IncMargin{-\parindent}

% Metadata Information
\acmVolume{V}
\acmNumber{N}
\acmArticle{A}
\acmYear{YYYY}
\acmMonth{0}

% Copyright
\setcopyright{rightsretained}

% DOI
\doi{0000001.0000001}

%ISSN
\issn{1234-56789}

% Document starts
\begin{document}

% Page heads
\markboth{K. Niemeyer et al.}{title TBD}

% Title portion
\title{title TBD}

\author{KYLE E. NIEMEYER
\affil{Oregon State University}
%ARFON M. SMITH
%\affil{GitHub Inc.}
DANIEL S. KATZ
\affil{University of Chicago \& Argonne National Laboratory}}
% NOTE! Affiliations placed here should be for the institution where the
%       BULK of the research was done. If the author has gone to a new
%       institution, before publication, the (above) affiliation should NOT be changed.
%       The authors 'current' address may be given in the "Author's addresses:" block (below).
%       So for example, Mr. Abdelzaher, the bulk of the research was done at UIUC, and he is
%       currently affiliated with NASA.

\begin{abstract}
\end{abstract}


%
% The code below should be generated by the tool at
% http://dl.acm.org/ccs.cfm
% Please copy and paste the code instead of the example below. 
%
\begin{CCSXML}
<ccs2012>
<concept>
<concept_id>10002951.10003227.10003233.10003449</concept_id>
<concept_desc>Information systems~Reputation systems</concept_desc>
<concept_significance>500</concept_significance>
</concept>
<concept>
<concept_id>10002951.10003227.10003233.10003597</concept_id>
<concept_desc>Information systems~Open source software</concept_desc>
<concept_significance>300</concept_significance>
</concept>
<concept>
<concept_id>10011007.10011074.10011111.10011696</concept_id>
<concept_desc>Software and its engineering~Maintaining software</concept_desc>
<concept_significance>500</concept_significance>
</concept>
<concept>
<concept_id>10010405.10010476.10010477</concept_id>
<concept_desc>Applied computing~Publishing</concept_desc>
<concept_significance>300</concept_significance>
</concept>
</ccs2012>
\end{CCSXML}

\ccsdesc[500]{Information systems~Reputation systems}
\ccsdesc[300]{Information systems~Open source software}
\ccsdesc[500]{Software and its engineering~Maintaining software}
\ccsdesc[300]{Applied computing~Publishing}

%
% End generated code
%

\terms{Software, Citation, Credit, Attribution}

\keywords{Software citation, Software credit}

\acmformat{Kyle E. Niemeyer, Arfon M.\ Smith, and Daniel M.\ Katz, 2016. Title TBD.}
% At a minimum you need to supply the author names, year and a title.
% IMPORTANT:
% Full first names whenever they are known, surname last, followed by a period.
% In the case of two authors, 'and' is placed between them.
% In the case of three or more authors, the serial comma is used, that is, all author names
% except the last one but including the penultimate author's name are followed by a comma,
% and then 'and' is placed before the final author's name.
% If only first and middle initials are known, then each initial
% is followed by a period and they are separated by a space.
% The remaining information (journal title, volume, article number, date, etc.) is 'auto-generated'.

\begin{bottomstuff}
This work is supported by the National Science Foundation, under ...

Author's addresses: K. E. Niemeyer, School of Mechanical, Industrial, and Manufacturing Engineering, Oregon State University; 
A. M. Smith,
GitHub, Inc.; 
D. M. Katz,
Computation Institute, University of Chicago \& Argonne National Laboratory.
\end{bottomstuff}

\maketitle


\section{Introduction}



\section{Related Work}

Ongoing efforts by the WSSSPE Software Credit Working Group and FORCE11 Software Citation Working Group

Codemeta for minimal metadata schemes in JSON and XML

transitive credit for via JSON-LD~\cite{Katz:2015aa}

\section{Challenges and Research Directions}

key challenges include
\begin{itemize}
\item standardize proper format for citing software in publications, and
\item identify necessary metadata associated with software for citation, 
\item develop infrastructure to support indexing of software citations along with (or complementary to) the existing publication citation ecosystem.
\end{itemize}

% Acknowledgments
%\begin{acks}
%\end{acks}

% Bibliography
\bibliographystyle{ACM-Reference-Format-Journals}
\bibliography{acmsmall-refs}
                             % Sample .bib file with references that match those in
                             % the 'Specifications Document (V1.5)' as well containing
                             % 'legacy' bibs and bibs with 'alternate codings'.
                             % Gerry Murray - March 2012

% History dates
\received{Month Year}{Month Year}{Month Year}

%% Electronic Appendix
%\elecappendix
%
%\medskip
%
%\section{This is an example of Appendix section head}

\end{document}
% End of v2-acmsmall-sample.tex (March 2012) - Gerry Murray, ACM


