% v2-acmsmall-sample.tex, dated March 6 2012
% This is a sample file for ACM small trim journals
%
% Compilation using 'acmsmall.cls' - version 1.3 (March 2012), Aptara Inc.
% (c) 2010 Association for Computing Machinery (ACM)
%
% Questions/Suggestions/Feedback should be addressed to => "acmtexsupport@aptaracorp.com".
% Users can also go through the FAQs available on the journal's submission webpage.
%
% Steps to compile: latex, bibtex, latex latex
%
% For tracking purposes => this is v1.3 - March 2012

\documentclass[prodmode,acmjdiq]{acmsmall} % Aptara syntax

\usepackage[utf8]{inputenc}
\usepackage[T1]{fontenc}
\usepackage[american]{babel}
\usepackage{textcomp}
\usepackage{csquotes}

\usepackage[breaklinks=true]{hyperref}
\usepackage{url}

\usepackage{booktabs,multicol}

\usepackage{color}
\usepackage[table]{xcolor}

\definecolor{lightgray}{gray}{0.9}
\definecolor{blueish}{rgb}{0.2,0.2,0.8}
\newcommand{\katznote}[1]{ {\textcolor{blueish} { ***DSK: #1 }}} % Dan
\newcommand{\knnote}[1]{ {\textcolor{orange} { ***KEN: #1 }}} % Kyle
\newcommand{\asnote}[1]{ {\textcolor{red} { ***AMS: #1 }}} %Arfon


% Package to generate and customize Algorithm as per ACM style
\usepackage[ruled]{algorithm2e}
\renewcommand{\algorithmcfname}{ALGORITHM}
\SetAlFnt{\small}
\SetAlCapFnt{\small}
\SetAlCapNameFnt{\small}
\SetAlCapHSkip{0pt}
\IncMargin{-\parindent}

% Metadata Information
\acmVolume{V}
\acmNumber{N}
\acmArticle{A}
\acmYear{YYYY}
\acmMonth{0}

% Copyright
\setcopyright{rightsretained}

% DOI
\doi{0000001.0000001}

%ISSN
\issn{1234-56789}

% Document starts
\begin{document}

% Page heads
\markboth{K.~Niemeyer et al.}{The challenge and promise of software citation for credit, identification, discovery, and reuse}

% Title portion
\title{The challenge and promise of software citation for credit, identification, discovery, and reuse}

\author{KYLE E. NIEMEYER
\affil{Oregon State University}
ARFON M. SMITH
\affil{GitHub, Inc.}
DANIEL S. KATZ
\affil{University of Illinois}}
% NOTE! Affiliations placed here should be for the institution where the
%       BULK of the research was done. If the author has gone to a new
%       institution, before publication, the (above) affiliation should NOT be changed.
%       The authors 'current' address may be given in the "Author's addresses:" block (below).
%       So for example, Mr. Abdelzaher, the bulk of the research was done at UIUC, and he is
%       currently affiliated with NASA.

\begin{abstract}
\end{abstract}


%
% The code below should be generated by the tool at
% http://dl.acm.org/ccs.cfm
% Please copy and paste the code instead of the example below.
%
\begin{CCSXML}
<ccs2012>
<concept>
<concept_id>10002951.10003227.10003233.10003449</concept_id>
<concept_desc>Information systems~Reputation systems</concept_desc>
<concept_significance>500</concept_significance>
</concept>
<concept>
<concept_id>10002951.10003227.10003233.10003597</concept_id>
<concept_desc>Information systems~Open source software</concept_desc>
<concept_significance>300</concept_significance>
</concept>
<concept>
<concept_id>10011007.10011074.10011111.10011696</concept_id>
<concept_desc>Software and its engineering~Maintaining software</concept_desc>
<concept_significance>500</concept_significance>
</concept>
<concept>
<concept_id>10010405.10010476.10010477</concept_id>
<concept_desc>Applied computing~Publishing</concept_desc>
<concept_significance>300</concept_significance>
</concept>
</ccs2012>
\end{CCSXML}

\ccsdesc[500]{Information systems~Reputation systems}
\ccsdesc[300]{Information systems~Open source software}
\ccsdesc[500]{Software and its engineering~Maintaining software}
\ccsdesc[300]{Applied computing~Publishing}

%
% End generated code
%

\terms{Documentation, Standardization}

\keywords{Software citation, Software credit, attribution}

\acmformat{Kyle E.\ Niemeyer, Arfon M.\ Smith, and Daniel S.\ Katz, 2016. The challenge and promise of software citation for credit, identification, discovery, and reuse.}
% At a minimum you need to supply the author names, year and a title.
% IMPORTANT:
% Full first names whenever they are known, surname last, followed by a period.
% In the case of two authors, 'and' is placed between them.
% In the case of three or more authors, the serial comma is used, that is, all author names
% except the last one but including the penultimate author's name are followed by a comma,
% and then 'and' is placed before the final author's name.
% If only first and middle initials are known, then each initial
% is followed by a period and they are separated by a space.
% The remaining information (journal title, volume, article number, date, etc.) is 'auto-generated'.

\begin{bottomstuff}
Work by K.\ E.\ Niemeyer was supported in part by the National Science Foundation (NSF) under grant ACI-1535065.
Work by D.\ S.\ Katz was supported by the NSF while working at the Foundation; any opinion, finding, and conclusions or recommendations expressed in this material are those of the author and do not necessarily reflect the views of the NSF. Some work by Katz was also done at the Computation Institute, University of Chicago \& Argonne National Laboratory.

Author's addresses: K.\ E.\ Niemeyer, School of Mechanical, Industrial, and Manufacturing Engineering, Oregon State University;
A.\ M.\ Smith,
GitHub, Inc.;
D.\ S.\ Katz,
University of Illinois;
emails: \href{mailto:kyle.niemeyer@oregonstate.edu}{kyle.niemeyer@oregonstate.edu}, \href{mailto:arfon@github.com}{arfon@github.com}, \href{mailto:d.katz@ieee.org}{d.katz@ieee.org}
\end{bottomstuff}

\maketitle

%%%%%%%%%%%%%%%%%%%%%%%%%
\section{Introduction}
%%%%%%%%%%%%%%%%%%%%%%%%%

Modern science and engineering depend on software, and much of it is not cited~\cite{Pan2015}.
In fact, a 2009 survey of scientists found that 91\% consider software important or very important to their research~\cite{Hannay:2009wp}.
The scientific community uses citation to acknowledge traditional research results published in archival journals and conferences, but no such accepted standard exists to credit the considerable efforts that go into software.

One method to increase the amount of software developed, shared, and credited is to treat a software release as a publication.
There is good evidence that academics respond to incentives, including interview and survey data saying that increased citation would drive increased software development and sharing~\cite{howison-herbsleb2011,huang2013}. Further, there is evidence that activity increases when outputs can be formally counted~\cite{mcnaught2015}. Finally, there is evidence from data publishing that direct citations to datasets have been successful, rather than relying on indirect citations through publications~\cite{belter2014}.

In addition, since research results, including data, now depend on the specific software used (e.g., version), proper citation---and the associated preservation---is necessary to ensure reproducibility~\cite{Sufi:2014en}.
\textbf{The main challenge of software citation is that we need a way to uniquely identify released\slash published software so that it can be cited by others who use it.}
We also need a process to curate and review software.

\knnote{introduce connection to data quality\slash information quality here}
Provenance of research results and data require, among other things, a record of the software used to generate or process that data~\cite{Sandve:2013gh,Wilson:2014aa}.
errors in software or variations due to environment~\cite{Morin:2012hz,Soergel:2015aa}


Different communities follow widely varying practices for citing software, with guidelines ranging from citations of an associated paper or the software itself via DOI~\cite{AAS:2016} to no policy at all, with many ad hoc practices in between~\cite{Howison2015}.
%Furthermore, some research communities have not yet adopted open-source mentalities regarding research software.
Questions also remain about the role of curating\slash reviewing software---if software will be cited in the same manner as a publication, is quality assessment needed (e.g., peer code review)?
If so, how, when, and by whom?

As a first step towards sustainable, reusable, and attributable software, efforts are underway to establish citation practices for software used in research.

%%%%%%%%%%%%%%%%%%%%%%%%%
\section{Related Work}
%%%%%%%%%%%%%%%%%%%%%%%%%

Multiple organizations, including the WSSSPE Software Credit~\cite{WSSSPE1,WSSSPE2,WSSSPE3} and FORCE11 Software Citation Working Groups, are working to standardize software citation practices.
Some of the observations and recommendations made here come from their efforts, which follow similar work by DataCite and the FORCE11~\citeN{DataCitation2014} to standardize research data citation practices.
Currently, services including Zenodo or fig\textbf{share} allow software developers to obtain DOIs for software; the former has a streamlined process~\cite{GitHubZenodo} to register DOIs for GitHub software releases.
The CodeMeta project~\cite{CodeMeta} is formalizing minimal metadata schemas in JSON and XML to connect existing software repositories, archives, and libraries (e.g., GitHub, fig\textbf{share}, Zenodo).

\knnote{mention workshops on software citation; University of Manchester (? can't actually find that one). Additional references are good.} 

\katznote{Some ideas: \\
\url{https://softwaredatacitation.org/} \\
\url{http://www.software.ac.uk/software-credit}  \\
\url{http://astronomy-software-index.github.io/2015-workshop/} \\
\url{https://geodynamics.org/cig/projects/saga/} \\
maybe \url{http://software.ac.uk/maintainable-software-practice-workshop}}
\knnote{Workshop reports} \cite{Sufi:2014en,Ahalt:2015ve}; Software Publishing Special Interest Group\cite{Allen:2015ub}

\katznote{Arfon, are the more that showed up in related work in the software citation activity?}

Complementary efforts to provide credit for software development in research also depend on standardized citation practices.
These include the transitive credit via JSON-LD scheme~\cite{Katz:2015aa} that assigns varying credit weights to contriponents---both contributors and research artifacts, including software and data---depending on their level of importance to the work.
The Project CRediT (contributor role taxonomy) scheme~\cite{project_credit} includes a ``software'' category for authors of a publication with roles dedicated to software development, maintenance, and\slash or testing.
The related PaperBadger project~\cite{PaperBadger} offers digital contributor badges associated with each role, connecting publications with people via DOI and ORCID, respectively.

\knnote{be more specific about related work targeting challenges}

%%%%%%%%%%%%%%%%%%%%%%%%%
\section{Challenges and Research Directions}
%%%%%%%%%%%%%%%%%%%%%%%%%

Key challenges and research directions (with possible solutions/methods) are shown in Table~\ref{tab:challenges}.

\rowcolors{2}{}{lightgray}
\begin{table}[htbp]%
\tbl{Key research challenges and possible solutions\slash methods\label{tab:challenges}}{%
\begin{tabular}{@{}p{0.38\linewidth}p{0.58\linewidth}@{}}
\toprule
\textbf{Key research Challenge} & \textbf{Possible solutions/methods} \\
\midrule

Identify necessary metadata associated with software for citation & The CodeMeta project~\cite{CodeMeta} is working to do this now \\

Standardize proper formats for citing software in publications & The FORCE11 Software Citation Working Group is currently working to define and gain community acceptance for software citation principles, after which (starting in April 2016) a follow-on group to implement the software citation principles will likely start, and part of the charge for this group will be to work with publishers on this challenge \\

Establish mechanisms for software to cite other software (i.e., dependencies) & Software publication as software papers will allow this.  For software that is directly published, this is an open challenge with no clear path forward \\

Develop infrastructure to support indexing of software citations along with (or complementary to) the existing publication citation ecosystem & The FORCE11 software citation implementation working group will also work on this, in collaboration with publishers and indexers \\

Determine standard practices for peer review of software & Professional societies and science communities need to determine how this will happen \\

Increase cultural acceptance of the concept of software as a digital product & Will happen over time; unclear how to accelarate \\ 
\bottomrule
\end{tabular}}
\end{table}

\knnote{suggest ``few possible answers or clear path that might be followed''}

For software to be cited, we recommend that metadata include software name, primary authors\slash contributors (name and ORCID), DOI or other unique identifier, location where the software has been published\slash archived (DOI\slash URL), and software dependencies (via DOI).
This information should be provided in a CITATION file, potentially in JSON or XML format (and with an appropriate metadata schema, e.g., DOAP) to replace or supplement plaintext and allow automatic processing.
Citations of software in publications should minimally include software name, primary authors, and DOI or location where the software was published\slash archived; however, individual citation formats will vary based on the particular style of journals, conferences, or professional societies.
Existing services such as Libraries.io~\cite{Libraries.io} and Depsy~\cite{Depsy} automate software dependency tracking; these could be harnessed to produce citation networks.
Technical solutions to these challenges exist; community acceptance is instead needed.

\knnote{add discussion of when software should be cited; mention F11 principles}
Question of when software should be cited; although no clear answer, in order to ensure reproducibility of research results, software should be cited if used directly and important to the results.
In other words, if using different software could produce different data or results, then the software used should be cited.

\katznote{done below - feel free to edit} The authors of this paper are currently leading
a working group under FORCE11 to develop principles for software citation~\cite{SCWG}.  The draft principles document~\cite{SoftwareCitation2016}, which is adapted from an earlier set of principles for data citation~\cite{DataCitation2014}, discusses
how the decision on what software to cite should be made.  It currently states:

\begin{quote}
What software should be cited is the decision of the author(s) of the research work, in the context of community norms and practices, and in most scientific communities, these are currently in flux.
In general, we believe that any software on which a research product directly depends should be cited, if it significantly contributes to the research product.
Again, the specific decision of what is significant must be made by the author(s) of the product.
However, an illustrative example is the use of Microsoft Excel in research.
We suggest that if Excel is used to simply store and plot data, it does not need to be cited, but if it is used for statistical analysis, it should be.  Similarly, general software for conducting library research (e.g., JSTOR Mobile App), writing research papers (e.g., \LaTeX), scientific presentations (e.g., Powerpoint) or communications (e.g., Skype) should not be cited.
This recommendation matches that of the Purdue Online Writing Lab: ``Do not cite standard office software (e.g. Word, Excel) or programming languages.  Provide references only for specialized software.''~\cite{powl-citing-software}

Note that some software which is or could be captured as part of data provenance may not be cited.
Citation is a record of software that is important to the research outcome, where provenance is a record of all steps (including software) used to generated particular data within the research process.
This implies that for a data research product, provenance data will include all cited software, but not necessarily vice versa.
Similarly, the software metadata that is recorded as part of data provenance should be a superset of the metadata recorded as part of software citation.
And the data recorded for reproducibility should also be a superset of the metadata recorded as part of software citation.
These statements may also be true for software products.
In general, we intend the software citation principles to cover the minimum of what is necessary for software citation for the purpose of software identification.
Other use cases (e.g., provenance, reproducibility) may lead to additional requirements (i.e., enhanced metadata).
\end{quote}



However, research and development efforts are needed to solve the remaining challenges.
These questions include: how can we cite closed-source\slash commercial software---can the above information be provided, even if the the software itself is not publicly preserved?
How will index services (e.g., Web of Science, Scopus, Google Scholar) fully index software, since software citations require indexing to create a citation network akin to publications to carry weight for academic credit and reputation?
How can publications indicate direct use of software for research in citations, where results would not be possible without efforts of software authors---should such a citation be ``weighted'' higher than others?
The \citeN{AAS:2016} suggests a new ``Software'' section below the acknowledgements; other approaches include machine-readable supplementary data~\cite{Katz:2015aa}.

Finally, open questions remain on whether citable software should go through peer review and, if so, how can this be implemented?
Should citable software itself follow the arXiv preprint model where releases are made available for users and the community to judge, or alternatively, the software paper model where ``advertising'' papers undergo peer review in a relevant community?


 Acknowledgments
\begin{acks}
The authors wish to thank the members of the SciSIP community and mailing list, in particular Chris Belter, James Howison, and Joshua Rosenbloom, for a useful discussion on incentives for software publications and how their impact has and could further be measured.
\end{acks}

% Bibliography
\bibliographystyle{ACM-Reference-Format-Journals}
\bibliography{acmsmall-refs}
                             % Sample .bib file with references that match those in
                             % the 'Specifications Document (V1.5)' as well containing
                             % 'legacy' bibs and bibs with 'alternate codings'.
                             % Gerry Murray - March 2012

% History dates
\received{Month Year}{Month Year}{Month Year}

%% Electronic Appendix
%\elecappendix
%
%\medskip
%
%\section{This is an example of Appendix section head}

\end{document}
% End of v2-acmsmall-sample.tex (March 2012) - Gerry Murray, ACM
