% v2-acmsmall-sample.tex, dated March 6 2012
% This is a sample file for ACM small trim journals
%
% Compilation using 'acmsmall.cls' - version 1.3 (March 2012), Aptara Inc.
% (c) 2010 Association for Computing Machinery (ACM)
%
% Questions/Suggestions/Feedback should be addressed to => "acmtexsupport@aptaracorp.com".
% Users can also go through the FAQs available on the journal's submission webpage.
%
% Steps to compile: latex, bibtex, latex latex
%
% For tracking purposes => this is v1.3 - March 2012

\documentclass[prodmode,acmjdiq]{acmsmall} % Aptara syntax

\usepackage[utf8]{inputenc}
\usepackage[T1]{fontenc}
\usepackage[american]{babel}
\usepackage{textcomp}
\usepackage{csquotes}

\usepackage[breaklinks=true]{hyperref}
\usepackage{url}

\usepackage{color}

\definecolor{blueish}{rgb}{0.2,0.2,0.8}
\newcommand{\note}[1]{ {\textcolor{blueish}    { ***Note:      #1 }}}


% Package to generate and customize Algorithm as per ACM style
\usepackage[ruled]{algorithm2e}
\renewcommand{\algorithmcfname}{ALGORITHM}
\SetAlFnt{\small}
\SetAlCapFnt{\small}
\SetAlCapNameFnt{\small}
\SetAlCapHSkip{0pt}
\IncMargin{-\parindent}

% Metadata Information
\acmVolume{V}
\acmNumber{N}
\acmArticle{A}
\acmYear{YYYY}
\acmMonth{0}

% Copyright
\setcopyright{rightsretained}

% DOI
\doi{0000001.0000001}

%ISSN
\issn{1234-56789}

% Document starts
\begin{document}

% Page heads
\markboth{K. Niemeyer et al.}{The challenge and promise of software citation for credit, identification, discovery, and reuse}

% Title portion
\title{The challenge and promise of software citation for credit, identification, discovery, and reuse}

\author{KYLE E. NIEMEYER
\affil{Oregon State University}
ARFON M. SMITH
\affil{GitHub Inc.}
DANIEL S. KATZ
\affil{University of Chicago \& Argonne National Laboratory}}
% NOTE! Affiliations placed here should be for the institution where the
%       BULK of the research was done. If the author has gone to a new
%       institution, before publication, the (above) affiliation should NOT be changed.
%       The authors 'current' address may be given in the "Author's addresses:" block (below).
%       So for example, Mr. Abdelzaher, the bulk of the research was done at UIUC, and he is
%       currently affiliated with NASA.

\begin{abstract}
\end{abstract}


%
% The code below should be generated by the tool at
% http://dl.acm.org/ccs.cfm
% Please copy and paste the code instead of the example below. 
%
\begin{CCSXML}
<ccs2012>
<concept>
<concept_id>10002951.10003227.10003233.10003449</concept_id>
<concept_desc>Information systems~Reputation systems</concept_desc>
<concept_significance>500</concept_significance>
</concept>
<concept>
<concept_id>10002951.10003227.10003233.10003597</concept_id>
<concept_desc>Information systems~Open source software</concept_desc>
<concept_significance>300</concept_significance>
</concept>
<concept>
<concept_id>10011007.10011074.10011111.10011696</concept_id>
<concept_desc>Software and its engineering~Maintaining software</concept_desc>
<concept_significance>500</concept_significance>
</concept>
<concept>
<concept_id>10010405.10010476.10010477</concept_id>
<concept_desc>Applied computing~Publishing</concept_desc>
<concept_significance>300</concept_significance>
</concept>
</ccs2012>
\end{CCSXML}

\ccsdesc[500]{Information systems~Reputation systems}
\ccsdesc[300]{Information systems~Open source software}
\ccsdesc[500]{Software and its engineering~Maintaining software}
\ccsdesc[300]{Applied computing~Publishing}

%
% End generated code
%

\terms{Documentation, Standardization}

\keywords{Software citation, Software credit, attribution}

\acmformat{Kyle E.\ Niemeyer, Arfon M.\ Smith, and Daniel S.\ Katz, 2016. The challenge and promise of software citation for credit, identification, discovery, and reuse.}
% At a minimum you need to supply the author names, year and a title.
% IMPORTANT:
% Full first names whenever they are known, surname last, followed by a period.
% In the case of two authors, 'and' is placed between them.
% In the case of three or more authors, the serial comma is used, that is, all author names
% except the last one but including the penultimate author's name are followed by a comma,
% and then 'and' is placed before the final author's name.
% If only first and middle initials are known, then each initial
% is followed by a period and they are separated by a space.
% The remaining information (journal title, volume, article number, date, etc.) is 'auto-generated'.

\begin{bottomstuff}
Work by D.\ S.\ Katz was supported by the National Science Foundation (NSF) while working at the Foundation; any opinion, finding, and conclusions or recommendations expressed in this material are those of the author and do not necessarily reflect the views of the NSF.
Some work by K.\ E.\ Niemeyer was supported by NSF grant ACI-1535065.

Author's addresses: K.\ E.\ Niemeyer, School of Mechanical, Industrial, and Manufacturing Engineering, Oregon State University; 
A.\ M.\ Smith,
GitHub, Inc.; 
D.\ S.\ Katz,
Computation Institute, University of Chicago \& Argonne National Laboratory.
\end{bottomstuff}

\maketitle

\note{the CFP asks us to address:\\
* What is an important data and information quality-related challenge
   facing organizations today?\\
 * Why is this important, and to which domains or disciplines?\\
 * How might this challenge be addressed?}
 
 \note{the 2-page limit does not seem firm, based on other published works - e.g. \url{http://dl.acm.org/citation.cfm?id=2786983}}
 
%%%%%%%%%%%%%%%%%%%%%%%%%
\section{Introduction}
%%%%%%%%%%%%%%%%%%%%%%%%%

Modern science and engineering depend on software. 
To acknowledge traditional research results published in archival journals and conferences, we use citation, but no such accepted standard exists to credit the considerable efforts that go into software.
One method to increase the amount of software developed and shared is to treat a software release as a publication. 
In addition, since research results now depend on the specific software used (e.g., version), proper citation---and the associated preservation---is necessary to ensure reproducibility.
\textbf{The main challenge of software citation is that we need a way to uniquely identify released (published) software so that it can be cited by others who use it.}
We also need a process to curate and review software.  

Different communities currently follow widely varying practices for citing software, with guidelines ranging from suggesting citations of associated software or the software directly via DOI~\cite{AAS:2016} to no policy at all, with many ad hoc practices in between~\cite{Howison2015}.
Furthermore, some research communities have not yet adopted open-source mentalities regarding research software.
Questions also remain about the role of curating\slash reviewing software---if software will be cited in the same manner as a publication, is quality assessment needed (e.g., should it go through peer code review?)
If so, how, when, and by whom?

As a first step towards sustainable, reusable, and attributable software, efforts are underway to establish citation practices for software used in research.

%%%%%%%%%%%%%%%%%%%%%%%%%
\section{Related Work}
%%%%%%%%%%%%%%%%%%%%%%%%%

Multiple groups, including the WSSSPE Software Credit~\cite{WSSSPE1,WSSSPE2,WSSSPE3} and FORCE11 Software Citation Working Groups, are working to standardize citation practices; some of the observations and recommendations made here come from these groups.
Our efforts follow similar work by DataCite and the FORCE11 Data Citation Synthesis Group~\cite{DataCitation2014} to standardize research data citation practices.
Currently, services including Zenodo or fig\textbf{share} allow software developers to obtain DOIs for software; the former has a streamlined process to register DOIs for GitHub software releases.
The CodeMeta project~\cite{CodeMeta} is formalizing minimal metadata schemes in JSON and XML to connect existing software repositories, archives, and libraries (e.g., GitHub, fig\textbf{share}, Zenodo.)

Complementary efforts to provide credit for software development in research also depend on standardized citation practices.
These include the transitive credit via JSON-LD scheme~\cite{Katz:2015aa} that assigns varying credit weights to contriponents---both contributors and research artifacts, including software and data---depending on their level of importance to the work.
The Project CRediT (contributor role taxonomy) scheme~\cite{project_credit} includes a ``software'' category for authors of a publication with roles dedicated to software development, maintenance, and\slash or testing.
The related PaperBadger project~\cite{PaperBadger} offers digital contributor badges associated with each role, connecting publications with people via DOI and ORCID, respectively.

%%%%%%%%%%%%%%%%%%%%%%%%%
\section{Challenges and Research Directions}
%%%%%%%%%%%%%%%%%%%%%%%%%

Key challenges are the need to:
\begin{itemize}
\item \note{added this, not sure if it really goes here} increase cultural acceptance of the concept of software as a digital product
\item standardize proper formats for citing software in publications,
\item identify necessary metadata associated with software for citation, 
\item establish mechanisms for software to cite other software (i.e., dependencies),
\item develop infrastructure to support indexing of software citations along with (or complementary to) the existing publication citation ecosystem, and
\item determine standard practices for peer review of software.
\end{itemize}
Although individual citation formats will vary based on the particular style of journals, conferences, or professional societies, citation of software in publications must minimally include software name, primary authors\slash contributors, DOI or other unique identifier, and location where the software has been published\slash archived (DOI or URL). 
The metadata needed for software to be cited includes software name, DOI, authors\slash contributors (with names and ORCIDs), software dependencies (in the form of DOIs), and any other people\slash artifacts that would be cited or acknowledged in a paper. \note{not sure about the previous two sentences?  explain more?  combine?}
This information should be provided in a CITATION file, potentially in JSON, XML, or DOAP (RDF\slash XML) formats---rather than or to supplement plaintext---to allow automatic processing\slash parsing.

However, additional research and development efforts are needed to solve the remaining challenges.
These questions include: how can we cite closed-source\slash commercial software?
Can the above information be provided, even if the the software itself is not available or preserved?
In order for software citations to carry weight akin to publications for academic credit and reputation, they require indexing to create a similar citation network (e.g., Web of Science, Scopus, Google Scholar).
How can publications indicate direct use of software for research in citations, where results would not be possible without efforts of software authors?
Should this citation be ``weighed'' higher than others?
The \citeN{AAS:2016} suggests a new ``Software'' section below the acknowledgements; other approaches include machine readable supplementary data~\cite{Katz:2015aa}.

Finally, open questions remain on whether citable software should go through peer review and, if so, how can this be implemented?
Instead, should citable software follow the arXiv preprint model where releases are made available for users and the community to judge, or alternatively the software paper model where ``advertising'' papers undergo peer review in a relevant community?


% Acknowledgments
%\begin{acks}
%\end{acks}

% Bibliography
\bibliographystyle{ACM-Reference-Format-Journals}
\bibliography{acmsmall-refs}
                             % Sample .bib file with references that match those in
                             % the 'Specifications Document (V1.5)' as well containing
                             % 'legacy' bibs and bibs with 'alternate codings'.
                             % Gerry Murray - March 2012

% History dates
\received{Month Year}{Month Year}{Month Year}

%% Electronic Appendix
%\elecappendix
%
%\medskip
%
%\section{This is an example of Appendix section head}

\end{document}
% End of v2-acmsmall-sample.tex (March 2012) - Gerry Murray, ACM


